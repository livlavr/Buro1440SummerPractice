\documentclass{article}
\usepackage{graphicx} % Required for inserting images
\usepackage[utf8]{inputenc}
\usepackage[russian]{babel}
\usepackage{xcolor}
\usepackage{makecell}
\usepackage{amsmath}
\usepackage{amssymb}
\usepackage{cancel}
\usepackage{listings}
\usepackage{tikz}
\linespread{1.5}
\definecolor{codegreen}{rgb}{0,0.6,0}
\definecolor{codegray}{rgb}{0.5,0.5,0.5}
\definecolor{codepurple}{rgb}{0.58,0,0.82}
\definecolor{backcolour}{rgb}{0.95,0.95,0.92}
\lstdefinestyle{myStyle}{
    backgroundcolor=\color{backcolour},
    commentstyle=\color{codegreen},
    keywordstyle=\color{magenta},
    numberstyle=\tiny\color{codegray},
    stringstyle=\color{codepurple},
    basicstyle=\ttfamily\footnotesize,
    breakatwhitespace=false,
    breaklines=true,
    keepspaces=true,
    numbers=left,
    numbersep=5pt,
    showspaces=false,
    showstringspaces=false,
    showtabs=false,
    tabsize=2,
}
\lstset{style=myStyle}
\usepackage[
    ignoreheadfoot, % set margins without considering header and footer
    top=2 cm, % seperation between body and page edge from the top
    bottom=2 cm, % seperation between body and page edge from the bottom
    left=2 cm, % seperation between body and page edge from the left
    right=2 cm, % seperation between body and page edge from the right
    footskip=0.5 cm, % seperation between body and footer
    % showframe % for debugging
]{geometry} % for adjusting page geometry

\begin{document}
\begingroup
    \centering
    \LARGE БЮРО 1440 (Задача 3)\\
    \LARGE Лаврущев Иван\\
    \large \today \\[0.5em]
\endgroup

\large
\section*{Доказательство периодичности чётности значений \(f(n)\)}

Рассмотрим:
$$f(n)=5 \cdot f(n-1) + f(n-2),$$
$$f(0)=1,f(1)=3$$
Посчитаем первые несколько членов \(f_n \% 2$\):
\begin{center}
    f_0 \equiv 1,\\
    f_1 \equiv 1,\\
    f_2 = f_1 + f_0 \equiv 1 + 1 \equiv 0\\
    f_3 = f_2 + f_1 \equiv 0 + 1 \equiv 1\\
    f_4 = f_3 + f_2 \equiv 1 + 0 \equiv 1\\
    f_5 = f_4 + f_3 \equiv 1 + 1 \equiv 0\\
    f_6 = f_5 + f_4 \equiv 0 + 1 \equiv 1\\
    f_7 = f_6 + f_5 \equiv 1 + 0 \equiv 1\\
    f_8 = f_7 + f_6 \equiv 1 + 1 \equiv 0\\
\end{center}
Заметим, что четные числа появляются на каждом индексе $3n + 2$, докажем более формально:
\begin{enumerate}
\item \textbf{Предположение индукции:} {\\f_{3n}   \equiv 0\\
                                          f_{3n+1} \equiv 1\\
                                          f_{3n+2} \equiv 1}

\item \textbf{База индукции:} {\\Верно для $n = 0, 1, 2$}
\item \textbf{Шаг индукции:} {\\Пусть для некоторого $k$ справедливо предположение индукции, тогда для следующей тройки чисел:\\\\
    f_{3k+3} = f_{3m} = f_{3k+2} + f_{3k+1} \equiv 0 + 1 \equiv 1,\\
    f_{3k+4} = f_{3m + 1} = f_{3k+3} + f_{3k+2} \equiv 1 + 0 \equiv 1,\\
    f_{3k+5} = f_{3m + 2} = f_{3k+4} + f_{3k+3} \equiv 1 + 1 \equiv 0}
\end{enumerate}\\
\textbf{ЧТД. Предположение индукции верно}
\\
\textbf{Вывод:} Четные значения встречаются каждый 3-ий индекс
\section*{Решение}
Таким образом, если мы хотим вычислить 40-й элемент массива $A$, то нас интересует $\frac{2}{3} \cdot x = 40$, $x = 60$ - номер (индексация с 1) четного числа в нужной нам тройки $\rightarrow$ номер нужного элемента = 59 (индексация с 1)\\

\begin{lstlisting}[language=C]
#include <stdio.h>
#include <stdlib.h>

int taskSolution();

int main() {
    int result = taskSolution();

    printf("%d\n", result);
}

int taskSolution() {
    int func_value_n_1 = 3;
    int func_value_n_2 = 1;
    int func_value     = 0;

    for (int n = 2; n < 59; n++) {
        func_value   = (5 * func_value_n_1 + func_value_n_2) % 100;
        func_value_n_2 = func_value_n_1;
        func_value_n_1 = func_value;
    }

    return func_value;
}
\end{lstlisting}
Последние две цифры числа - 11\\
Значит правильный ответ: \textbf{184153577162052268122747461393215875186211}\\
\textbf{Ответ:} 184153577162052268122747461393215875186211

\end{document}
